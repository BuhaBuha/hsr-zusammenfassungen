% /*
%  * ----------------------------------------------------------------------------
%  * "THE BEER-WARE LICENSE" (Revision 42):
%  * <michi.wieland@hotmail.com> wrote this file. As long as you retain this notice you
%  * can do whatever you want with this stuff. If we meet some day, and you think
%  * this stuff is worth it, you can buy me a beer in return. Michael Wieland
%  * ----------------------------------------------------------------------------
%  */

\documentclass[
a4paper,
oneside,
10pt,
fleqn,
headsepline,
toc=listofnumbered, 
bibliography=totocnumbered]{scrartcl}

% deutsche Trennmuster etc.
\usepackage[T1]{fontenc}
\usepackage[utf8]{inputenc}
\usepackage[english, ngerman]{babel} % \selectlanguage{english} if  needed
\usepackage{lmodern} % use modern latin fonts

% Custom commands
\newcommand{\AUTHOR}{Michael Wieland}
\newcommand{\INSTITUTE}{Hochschule für Technik Rapperswil}
\newcommand{\GITHUB}{https://github.com/michiwieland/hsr-zusammenfassungen}
\newcommand{\LICENSEURL}{https://en.wikipedia.org/wiki/Beerware}
\newcommand{\LICENSE}{
"THE BEER-WARE LICENSE" (Revision 42):
<michi.wieland@hotmail.com> wrote this file. As long as you retain this notice you
can do whatever you want with this stuff. If we meet some day, and you think
this stuff is worth it, you can buy me a beer in return. Michael Wieland	
}

% Jede Überschrift 1 auf neuer Seite
\let\stdsection\section
\renewcommand\section{\clearpage\stdsection}

% Multiple Authors
\usepackage{authblk}

% Include external pdf
\usepackage{pdfpages}

% Layout / Seitenränder
\usepackage{geometry}

% Inhaltsverzeichnis
\usepackage{makeidx} 
\makeindex

\usepackage{url}
\usepackage[pdfborder={0 0 0}]{hyperref}
\usepackage[all]{hypcap}
\usepackage{hyperxmp} % for license metadata

% Glossar und Abkürzungsverzeichnis
\usepackage[acronym,toc,nopostdot]{glossaries}
\glossarystyle{altlist}
\usepackage{xparse}
\DeclareDocumentCommand{\newdualentry}{ O{} O{} m m m m } {
	\newglossaryentry{gls-#3}{
		name={#4 : #5},
		text={#5\glsadd{#3}},
		description={#6},
		#1
	}
	\makeglossaries
	\newacronym[see={[Siehe:]{gls-#3}},#2]{#3}{#4}{#5\glsadd{gls-#3}}
}
\makeglossaries

% Mathematik
\usepackage{amsmath}
\usepackage{amssymb}
\usepackage{amsfonts}
\usepackage{enumitem}

% Images
\usepackage{graphicx}
\graphicspath{{images/}} % default paths

% Boxes
\usepackage{fancybox}

%Tables
\usepackage{tabu}
\usepackage{booktabs} % toprule, midrule, bottomrule
\usepackage{array} % for matrix tables

% Multi Columns
\usepackage{multicol}

% Header and footer
\usepackage{scrlayer-scrpage}
\setkomafont{pagehead}{\normalfont}
\setkomafont{pagefoot}{\normalfont}
\automark*{section}
\clearpairofpagestyles
\ihead{\headmark}
\ohead{\AUTHOR}
\cfoot{\pagemark}

% Pseudocode
\usepackage{algorithmic}
\usepackage[linesnumbered,ruled]{algorithm2e}

% Code Listings
\usepackage{listings}
\usepackage{color}
\usepackage{beramono}

\definecolor{bluekeywords}{rgb}{0,0,1}
\definecolor{greencomments}{rgb}{0,0.5,0}
\definecolor{redstrings}{rgb}{0.64,0.08,0.08}
\definecolor{xmlcomments}{rgb}{0.5,0.5,0.5}
\definecolor{types}{rgb}{0.17,0.57,0.68}

\lstdefinestyle{visual-studio-style}{
	language=[Sharp]C,
	columns=flexible,
	showstringspaces=false,
	basicstyle=\footnotesize\ttfamily, 
	commentstyle=\color{greencomments},
	morekeywords={partial, var, value, get, set},
	keywordstyle=\bfseries\color{bluekeywords},
	stringstyle=\color{redstrings},
	breaklines=true,
	breakatwhitespace=true,
	tabsize=4,
	numbers=left,
	numberstyle=\tiny\color{black},
	frame=lines,
	showspaces=false,
	showtabs=false,
	escapeinside={£}{£},
}

\definecolor{DarkPurple}{rgb}{0.4, 0.1, 0.4}
\definecolor{DarkCyan}{rgb}{0.0, 0.5, 0.4}
\definecolor{LightLime}{rgb}{0.3, 0.5, 0.4}
\definecolor{Blue}{rgb}{0.0, 0.0, 1.0}

\lstdefinestyle{cevelop-style}{
	language=C++,  
	columns=flexible,
	showstringspaces=false,     
	basicstyle=\footnotesize\ttfamily, 
	keywordstyle=\bfseries\color{DarkPurple},
	commentstyle=\color{LightLime},
	stringstyle=\color{Blue}, 
	escapeinside={£}{£}, % latex scope within code      
	breaklines=true,
	breakatwhitespace=true,
	showspaces=false,
	showtabs=false,
	tabsize=4,
	morekeywords={include,ifndef,define},
	numbers=left,
	numberstyle=\tiny\color{black},
	frame=lines,
}

\lstdefinestyle{eclipse-style}{
	language=Java,  
	columns=flexible,
	showstringspaces=false,     
	basicstyle=\footnotesize\ttfamily, 
	keywordstyle=\bfseries\color{DarkPurple},
	commentstyle=\color{LightLime},
	stringstyle=\color{Blue}, 
	escapeinside={£}{£}, % latex scope within code      
	breaklines=true,
	breakatwhitespace=true,
	showspaces=false,
	showtabs=false,
	tabsize=4,
	morekeywords={length},
	numbers=left,
	numberstyle=\tiny\color{black},
	frame=lines,
}
\lstset{style=eclipse-style}



% Theorems \begin{mytheo}{title}{label}
\usepackage{tcolorbox}
\tcbuselibrary{theorems}
\newtcbtheorem[number within=section]{definiton}{Definition}%
{fonttitle=\bfseries}{def}
\newtcbtheorem[number within=section]{remember}{Merke}%
{fonttitle=\bfseries}{rem}
\newtcbtheorem[number within=section]{hint}{Hinweis}%
{fonttitle=\bfseries}{hnt}

% Dokumentinformationen
\newcommand{\SUBJECT}{Zusammenfassung}
\newcommand{\TITLE}{Math Essentials}

% pdf metadata
\hypersetup{
	pdfauthor={\AUTHOR},
	pdftitle={\SUBJECT \TITLE},
	pdfcopyright={\LICENSE},
	pdflicenseurl={\LICENSEURL}
}

\begin{document}
	
% Front page
\title{\TITLE}
\subject{\SUBJECT}
\author{\AUTHOR}
\affil{\INSTITUTE}
\date{\today}
\maketitle

\vfill

% Participate
\paragraph{Mitmachen} \hfill \\
Falls Du an diesem Dokument mitarbeiten willst, kannst Du das Dokument
auf GitHub unter \url{\GITHUB} forken.

% Licence
\paragraph{Lizenz} \hfill \\
\LICENSE

% Table of contents
\tableofcontents


% Glossar and acronyms (if included \loadglsentries{glossar})
\printglossary[type=\acronymtype]
\printglossary
\glsaddall


\section{Klammerregeln}
\subsection{Addition und Subtraktion}
Steht ein Pluszeichen vor einer Klammer, so kann diese einfach weggelassen werden.
\begin{align*}
a+ (b - c) = a + b - c
\end{align*}
Steht hingegen ein Minuszeichen vor der Klammer müssen alle Vorzeichen in der Klammer umgedreht werden.
\begin{align*}
a - (b - c) = a - b + c
\end{align*}

\section{Bruchrechnen}
\subsection{Terminologie}
\begin{description}
	\item[Zähler] Der Zähler ist oberhalb des Bruchstriches und wird auch Divident genannt
	\item[Nenner] Der Nenner ist unterhalb des Bruchstriches und wird auch Divisor genannt
	\item[Teilen durch 0] Das Teilen durch 0 ist verboten
	\item[echte Brüche] Der Zähler ist kleiner als der Nenner $\frac{4}{13}$
 	\item[unechte Brüche] Der Zähler ist grösser als der Nenner $\frac{19}{11}$
 	\item[Stammbruch] Stammbrüche haben als Zähler eine 1. $\frac{1}{2}$ oder $\frac{1}{12}$
 	\item[Kehrwert] Beim Kehrwert wird der Zähler und der Nenner miteinander vertuascht.
\end{description}

\subsection{Vorzeichen}
Das Vorzeichen kann im Zähler, im Nenner oder vor dem Bruch geschrieben werden
\begin{align*}
\frac{-a}{b} = \frac{a}{-b} = -\frac{a}{b}
\end{align*}

\subsection{Addition und Subtraktion}
Additionen und Subtraktionen sind nur dann möglich, wenn der Nenner der beteiligten Brüche gleich sind. Dazu müssen die Brüche meist gleichnamig gemacht werden. (kgV bilden) 
\begin{align*}
\frac{a}{b} \pm \frac{c}{d} = \frac{a\cdot d \pm b \cdot b}{b \cdot d}
\end{align*}

\subsection{Multiplikation}
Bei der Multiplikation wird der Zähler und der Nenner multipliziert.
\begin{align*}
\frac{a}{b} \cdot \frac{c}{d} = \frac{a \cdot c}{b \cdot d}
\end{align*}

\subsection{Erweitern und Kürzen}
Beim erweitern und kürzen wird der Bruch um einen Wert c ergänzt oder gekürtzt. Beim Erweitern ändert sich der Wert des Bruches nicht, da man den Wert direkt wieder kürzen könnte. Manchmal ist es aber hilfreich einen Bruch so zu erweitern, dass weiter umgeformt werden kann.
\begin{align*}
\frac{a}{b} = \frac{a \cdot c}{b \cdot c} 
\end{align*}

\subsection{Doppelbrüche}
\begin{align*}
	\frac{\frac{a}{b}}{\frac{c}{d}} &= \frac{a \cdot d}{b \cdot c} \\
	\frac{\frac{a}{b}}{c} &= \frac{a}{a \cdot c} \\
\end{align*}

\subsection{Bruch als Potenz}
\begin{align*}
\frac{1}{a} = a^{-1}
\end{align*}

\section{Wurzeln}
\subsection{Grundgesetze}
\begin{align*}
\sqrt[n]{1} = 1 \\
\sqrt[n]{0} = 0
\end{align*}

\subsection{Addition und Subtraktion}
Wurzeln über einer Addition oder Subtraktion dürfen nicht einzeln betrachtet werden
\begin{align*}
\sqrt{a \pm b} \neq \sqrt{a} \pm \sqrt{b}
\end{align*}

\subsection{Multiplikation und Division}
Brüche
\begin{align*}
\sqrt{a \cdot b} &= \sqrt{a} \cdot \sqrt{b} \\
\sqrt{\frac{a}{b}} &= \frac{\sqrt{a}}{\sqrt{b}} 
\end{align*}

\subsubsection{Verschachtelungen}
\begin{align*}
\sqrt[n]{\sqrt[m]{a}} = \sqrt[n \cdot m]{a}
\end{align*}

\subsubsection{Wurzel als Potenz}
\begin{align*}
\sqrt[n]{a} &= a^{\frac{1}{n}} \\
\sqrt[n]{a^m} &= a^{\frac{m}{n}} \\
\sqrt[-n]{a} &= \frac{1}{\sqrt[n]{a}}
\end{align*}

\section{Potenzen}
\subsection{Grundgesetze}
\begin{align*}
a^1 &= a \\
1^n &= 1 \\
a^0 &= 1 | a \neq 0 \\
0^0 &= undefiniert
\end{align*}

\subsection{Multiplikation und Division}
\begin{align*}
a^n \cdot b^n &= (a \cdot b)^n \\
a^n \cdot a^m &= a^{n+m} \\
\frac{a^n}{b^n} &= \left(\frac{a}{b}\right)^n \\
\frac{a^n}{a^m} &= a^{n-m} 
\end{align*}

\subsection{Mehrere Potenzen}
\begin{align*}
\left(a^n\right)^m = a^{n \cdot m}
\end{align*}

\subsection{Negative Potenzen}
\begin{align*}
a^{-n} = \frac{1}{a^n} = \left(\frac{1}{n}\right)^n
\end{align*}


\section{Logarithmen}
\subsection{Grundgesetze}
\begin{align*}
\log_a(a^b) = b \\
\log_a(1) = 0 \\
\log_a(a) = 1
\end{align*}


\subsection{Addition und Subtraktion}
Bei der Addition und Subtraktion können die Logarithmen nicht vereinfacht werden.
\begin{align*}
	\log_a(b + c) &= \log_a(b + c) \\
	\log_a(b - c) &= \log_a(b - c) \\
\end{align*}

\subsection{Multiplikation und Division}
\begin{align*}
	\log_a(b \cdot c) &= \log_a(b) + \log_a(c) \\
	\log_a\left(\frac{b}{c}\right) &= \log_a(b) - \log_a(c) \\
\end{align*}


\subsection{Potenzen}
\begin{align*}
a^{log_a(b)} &= b \\
\log_a(b^c) &= c \cdot \log_a(b)
\end{align*}


\section{Binomische Formeln}
\paragraph{Plus-Form}
\begin{align*}
(a + b)^2 &= (a + b) (a + b) \\
		  &= a^2 + 2ab + b^2
\end{align*}

\paragraph{Minus-Form}
\begin{align*}
(a - b)^2 &= (a - b)(a - b) \\
		  &= a^2 - 2ab + b^2
\end{align*}


\paragraph{Plus-Minus-Form}
\begin{align*}
(a+b)(a-b) &= a^2 + ba - ab - b^2 \\
		   &= a^2 - b^2
\end{align*}

\subsection{Pascalsche Dreieck}
Das pascalsche Dreieck ist eine geometrische Darstellung der Binomialkoeffizienten. Mit ihm lassen sich binomische Formen vom Grad N bestimmen.

\hfill \\ 

\begin{table}[h]
	\begin{tabular}{>{$N=}l<{$\hspace{12pt}}*{13}{c}}
		0 &&&&&&&1&&&&&&\\
		1 &&&&&&1&&1&&&&&\\
		2 &&&&&1&&2&&1&&&&\\
		3 &&&&1&&3&&3&&1&&&\\
		4 &&&1&&4&&6&&4&&1&&\\
		5 &&1&&5&&10&&10&&5&&1&\\
		6 &1&&6&&15&&20&&15&&6&&1
	\end{tabular}
	\caption{Pascalsche Dreieck}
	\label{tbl:pascals-triangle}
\end{table}


\begin{align*}
(a + b)^2 &= a^2 + 2ab + b^2 \\
(a + b)^3 &= a^3 + 3a^2b + 3ab^2 + b^3 \\
(a + b)^4 &= a^4 + 4a^3b + 6a^2b^2 + 4ab^3 + b^4 \\
(a + b)^5 &= a^5 + 5a^4b + 10a^3b^2 + 10a^2b^3 + 5ab^4 +b^5 \\
(a + b)^6 &= a^6 + 6a^5b + 15a^4b^2 + 20a^3b^3 + 15a^2b^4 + 6ab^5 + b^6 \\
\end{align*}

\section{Sinus, Kosinus und Tangens}
\subsection{Fundamentalbeziehungen}
\begin{align*}
\tan(x) &= \frac{\sin(x)}{\cos(x)} \\
1 &= \cos^2(x) + \sin^2(x)
\end{align*}

%TODO Quadrantenbeziehungen, 

\appendix

% Code Listings
\lstlistoflistings

% List of figures
\listoffigures

% List of tables
\listoftables

% Bibliography
\bibliographystyle{plain} 
\bibliography{literatur}

\end{document}